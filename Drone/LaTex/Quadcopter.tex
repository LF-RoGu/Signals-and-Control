\documentclass[a4paper,12pt]{report}
\usepackage{amsmath}
\usepackage{graphicx}
\usepackage{float}
\usepackage{hyperref}
\usepackage{caption}
\usepackage{subcaption}

\begin{document}

\title{Quadcopter Control Model: Mathematical Model and Simulation}
\author{Luis Fernando Rodriguez Gutierrez}
\date{\today}
\maketitle

\tableofcontents

\chapter{Introduction}
Autonomous operation of aerial vehicles relies heavily on onboard stabilization and trajectory tracking capabilities. 
Ensuring that these systems achieve stable flight requires significant effort, particularly at smaller scales where environmental effects and sensor noise play a larger role. 
Since the number and complexity of applications for UAVs continue to grow, the control techniques used must also evolve to provide better performance and increased versatility.
Historically, simplistic linear control techniques were employed for computational ease and stable hover flight. 
However, with advancements in modeling techniques and faster onboard computational capabilities, comprehensive nonlinear control methods have become achievable in real-time applications.
Nonlinear methodologies promise to significantly enhance the performance of UAVs, making them more robust and capable of handling complex tasks.

In this context, modeling drones using a 6 Degrees of Freedom (6 DOF) framework becomes essential. 
The 6 DOF model captures both the translational and rotational dynamics of the drone, providing a comprehensive mathematical framework to describe its behavior. 
The model includes all six independent motions that a drone can undergo: translation along the x, y, and z axes (forward/backward, left/right, upward/downward) and rotation around the x, y, and z axes (roll, pitch, and yaw).

The control strategies developed for these 6 DOF models must address the challenges posed by the UAV's small size and the associated environmental and sensor-related issues. 
Various control strategies, including both linear and nonlinear methods, are explored to stabilize and control the drone effectively. The integration of these advanced control techniques, alongside accurate modeling of the drone's dynamics, is crucial for achieving reliable and efficient operation of sUAVs in real-world environments.

\chapter{Quadcopter Dynamics}
The modeling and control of drones, particularly quadcopters, is a challenging yet vital area of research in the field of aerospace and robotics. 
Drones are complex systems characterized by their ability to maneuver in three-dimensional space, which is best represented using a 6 Degrees of Freedom (6 DOF) model. 
This model captures both the translational and rotational dynamics of the drone, providing a comprehensive mathematical framework to describe its behavior.

A drone's dynamics are described using the principles of Newtonian mechanics and rigid body dynamics. These principles are applied to two separate but interconnected motions:
\begin{itemize}
    \item Translational motion: Describes the movement of the drone along the x, y, and z axes.
    \item Rotational motion: Describes the changes in the drone’s orientation (roll, pitch, and yaw).
\end{itemize}

In the 6 DOF model, the translational and rotational dynamics are coupled. For example, a change in the drone’s pitch or roll angle (rotation) will alter the direction of the thrust force, which in turn affects the translational motion.

\section{Translational Motion}
The translational dynamics describe the drone’s movement along the x, y, and z axes of the inertial frame. These dynamics are governed by Newton’s second law of motion:
\[
m \ddot{\mathbf{r}} = \mathbf{F}_{\text{thrust}} + \mathbf{F}_{\text{gravity}} + \mathbf{F}_{\text{external}}
\]
In the body frame, the total thrust is generated by the drone's rotors and is distributed in all directions depending on the drone’s orientation.
\begin{itemize}
    \item \(m\) is the mass of the drone.
    \item \(\mathbf{r} = [x, y, z]^\text{T}\) is the position vector of the drone.
    \item \(\mathbf{F}_{\text{thrust}}\) represents the thrust forces generated by the drone's motors.
    \item \(\mathbf{F}_{\text{gravity}}\) represents the gravitational force acting on the drone.
    \item \(\mathbf{F}_{\text{external}}\) represents any external forces, such as wind or drag.
\end{itemize}

\section{Rotational Motion}
Rotational dynamics describe how the drone's orientation changes over time due to the torques applied by its motors. These dynamics are described using Euler’s rotational equations:
\begin{itemize}
    \item \(\mathbf{I}\) is the inertia matrix of the drone.
    \item \(\boldsymbol{\omega} = [p, q, r]^\text{T}\) is the angular velocity vector in the body frame.
    \item \(\boldsymbol{\tau} = [\tau_{\phi}, \tau_{\theta}, \tau_{\psi}]^\text{T}\) is the vector of torques around the x, y, and z axes (roll, pitch, and yaw).
\end{itemize}


\section{Degrees Of Freedom (DOF)}
The 6 Degrees of Freedom refer to the two types of movement that an object can exert on itself, translational and rotational. 
These can be described as six independent motions that a drone can undergo:
\begin{itemize}
    \item Translation along the x-axis (forward/backward motion).
    \item Translation along the y-axis (left/right motion).
    \item Translation along the z-axis (upward/downward motion).
    \item Rotation around the x-axis (roll).
    \item Rotation around the y-axis (pitch).
    \item Rotation around the z-axis (yaw).
\end{itemize}

\subsection{Equations of Motion}

\subsubsection{Translational Motion:}
\begin{align*}
\ddot{x} &= \frac{F_T}{m} (\cos{\phi} \sin{\theta} \cos{\psi} + \sin{\phi} \sin{\psi}) \\
\ddot{y} &= \frac{F_T}{m} (\cos{\phi} \sin{\theta} \sin{\psi} - \sin{\phi} \cos{\psi}) \\
\ddot{z} &= \frac{F_T}{m} (\cos{\phi} \cos{\theta}) - g
\end{align*}
Where \(\mathbf{F}_{\text{T}}\) is the total thrust generated by the motors, and \(\mathbf{g}\) is the gravitational constant.

\subsubsection{Rotational Motion:}
\begin{align*}
\tau_x &= I_x \dot{p} + (I_z - I_y) qr + I_{xz}(q^2 - r^2) \\
\tau_y &= I_y \dot{q} + (I_x - I_z) rp + I_{xz}(r^2 - p^2) \\
\tau_z &= I_z \dot{r} + (I_y - I_x) pq + I_{xz}(p^2 - q^2)
\end{align*}

\section{Thrust and Motor Dynamics}
In the case of quadcopters, the thrust generated by each motor plays a critical role in controlling both the translational and rotational dynamics. By adjusting the speed of each motor, the drone can achieve different maneuvers:

\begin{itemize}
    \item \textbf{Roll} is controlled by changing the thrust between the left and right motors.
    \item \textbf{Pitch} is controlled by changing the thrust between the front and rear motors.
    \item \textbf{Yaw} is controlled by the differential torque produced by motors rotating in opposite directions (due to conservation of angular momentum).
\end{itemize}

Each motor's thrust can be modeled as a function of its rotational speed:

\[
F_T = k_T \cdot \omega^2
\]

Where:
\begin{itemize}
    \item $F_T$ is the thrust generated by the motor.
    \item $k_T$ is a thrust constant that depends on the motor and propeller characteristics.
    \item $\omega$ is the angular speed of the motor.
\end{itemize}


\section{Coordinate Systems}
For any type of Autonomous vehicle a coordinate system is needed.
This so the vehicle can calculate the current position and where to go next.
The most commonly used for drones is the two reference frame is used for drones.
\begin{itemize}
    \item Inertial Frame (Earth Frame): This is a fixed, world-based reference frame used to describe the absolute position and orientation of the drone.
    \item Body Frame: This is a frame fixed to the drone, with the origin at the drone’s center of mass. Movements and rotations are described relative to this body frame.
\end{itemize}

\section{Euler Angles}
Euler angles are crucial in describing the orientation of the quadcopter in 3D space.
These angles (yaw, pitch, roll) represent the rotation in the axes of a coordinated system, transforming the orientation of the drone from the body frame to the inertial frame.
In this system:
\begin{itemize}
    \item The X-axis points forward (toward the nose of the vehicle).
    \item The Y-axis points to the right (toward the starboard side).
    \item The Z-axis points downward (toward the keel of the vehicle).
\end{itemize}

In aeronautics, the NED (North-East-Down) system is commonly used to describe orientation relative to the vehicle. 
The rotational velocity of the vehicle is described by the rates of change around these axes, denoted as \(p\), \(q\), and \(r\). 
Using this velocities of the vehicle we can calculate the orientation of the vehicle employing the Euler’s angles. These can be given by the following equations:
\[
    R_z(\psi) = \begin{bmatrix}
        \cos \psi & -\sin \psi & 0 \\
        \sin \psi & \cos \psi  & 0 \\
        0         & 0          & 1
    \end{bmatrix}, \quad
    R_y(\theta) = \begin{bmatrix}
        \cos \theta  & 0 & \sin \theta \\
        0            & 1 & 0           \\
        -\sin \theta & 0 & \cos \theta
    \end{bmatrix}, \quad
    R_x(\phi) = \begin{bmatrix}
        1 & 0         & 0          \\
        0 & \cos \phi & -\sin \phi \\
        0 & \sin \phi & \cos \phi
    \end{bmatrix}
\]

The orientation of the local system relative to the global system is described by the Euler angles: roll (\(\phi\)), pitch (\(\theta\)), and yaw (\(\psi\)).

The kinematic equations for the Euler angles describe how the angular velocities p, q, and r (around the X, Y, and Z axes, respectively) relate to the rates of change of the Euler angles:
\begin{equation}
    \begin{bmatrix}
    \dot{\phi} \\
    \dot{\theta} \\
    \dot{\psi}
    \end{bmatrix}
    =
    \begin{bmatrix}
    1 & \sin(\phi) \tan(\theta) & \cos(\phi) \tan(\theta) \\
    0 & \cos(\phi) & -\sin(\phi) \\
    0 & \sin(\phi)/\cos(\theta) & \cos(\phi)/\cos(\theta)
    \end{bmatrix}
    \begin{bmatrix}
    p \\
    q \\
    r
    \end{bmatrix}
\end{equation}
    
These relationships are vital in understanding how the drone's orientation evolves over time under the influence of various torques applied by the rotors.

In practical control systems, such as those implemented with PID controllers, Euler angles are used to stabilize and control the drone's orientation. 
The controller adjusts the motor speeds to correct deviations from desired roll, pitch, and yaw angles, ensuring that the drone maintains its intended orientation.
\chapter{State-Space Representation}

\section{3 DOF State-Space Representation}
To convert the equations of motion from the 3 DOF model into a state-space representation, we define the state vector \(\mathbf{x}\) and the input vector \(\mathbf{u}\). The state-space equations take the form:

\[
    \dot{\mathbf{x}} = A\mathbf{x} + B\mathbf{u}
\]
\[
    \mathbf{y} = C\mathbf{x} + D\mathbf{u}
\]

Where:
\[
    \mathbf{x} = \begin{bmatrix} x \\ y \\ \psi \\ \dot{x} \\ \dot{y} \\ \dot{\psi} \end{bmatrix}, \quad \mathbf{u} = \begin{bmatrix} u_x \\ u_y \\ u_\psi \end{bmatrix}
\]

The matrices \(A\), \(B\), \(C\), and \(D\) are derived based on the equations of motion.

\section{6 DOF State-Space Representation}
For the 6 DOF model, the state vector \(\mathbf{x}\) includes all six degrees of freedom (position and orientation, as well as their derivatives). The state-space equations are extended to include these dynamics:

\[
    \dot{\mathbf{x}} = A\mathbf{x} + B\mathbf{u}
\]
\[
    \mathbf{y} = C\mathbf{x} + D\mathbf{u}
\]

Where:
\[
    \mathbf{x} = \begin{bmatrix} x \\ y \\ z \\ \phi \\ \theta \\ \psi \\ \dot{x} \\ \dot{y} \\ \dot{z} \\ p \\ q \\ r \end{bmatrix}, \quad \mathbf{u} = \begin{bmatrix} u_x \\ u_y \\ u_z \\ u_\phi \\ u_\theta \\ u_\psi \end{bmatrix}
\]

Again, the matrices \(A\), \(B\), \(C\), and \(D\) are derived from the equations of motion, accounting for all six degrees of freedom.

\chapter{Control Strategies}

Various control strategies can be applied to stabilize and control drones modeled using 6 DOF:

\begin{itemize}
    \item \textbf{PID Control}: Simple but effective for basic stabilization and trajectory following.
    \item \textbf{Linear Quadratic Regulator (LQR)}: A more advanced method that optimizes the control inputs by minimizing a cost function.
    \item \textbf{Model Predictive Control (MPC)}: Predicts the future states of the system and optimizes control actions over a finite horizon.
    \item \textbf{Adaptive Control}: Adjusts the control parameters in real time to compensate for system uncertainties or disturbances.
\end{itemize}

\end{document}
