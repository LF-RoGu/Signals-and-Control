\documentclass[a4paper,12pt]{report}
\usepackage{amsmath}
\usepackage{graphicx}
\usepackage{float}
\usepackage{hyperref}
\usepackage{caption}
\usepackage{subcaption}

\begin{document}

\title{Quadcopter Control Model: Mathematical Model and Simulation}
\author{Luis Fernando Rodriguez Gutierrez}
\date{\today}
\maketitle

\tableofcontents

\chapter{Introduction}

\section{Background}
The design of a quadcopter is an interesting design as being a system that only requires three components that apply a force to this system.
It is able to move fully in a three-dimentional plane. However this is an unique feature of a quadcopter, as result of this difference of input versus outputs, the system becomes underactuated, meaning that we have less inputs in the system than outputs.
A negative effect of this underactuated system is that this system by nature can not reach a natural stable state. Meaning that unless the system is at rest, this system will not reach a stable state without the help of a controller.
\section{Purpose}
The final purpose of this paper is to provide an understanding on how Quadcopters work, as well as introduce a few of the most common control models that are used in the market.
While providing how the space state system was obtained, and a simulation on how this system will be working. While at this moment an IMU will not be implemented, the controller will only work in regards of the position of the system as a whole, while disregarding the yaw, pitchand roll.

\section{Limitations}
\section{Overview}
This document presents the modeling and control strategies for a quadcopter drone with both 3 Degrees of Freedom (DOF) and 6 Degrees of Freedom (DOF).
The drone's dynamics are derived using Newton's and Euler's laws.
A linearized version of the model is obtained to facilitate control design.
Various control strategies are explored to achieve stable and accurate flight control.

\chapter{Quadrotor Dynamics}

\section{Euler Angles}
For understanding a little more how the system, we will use a little this euler angles and use as well some of the terminology that it is used in aeronautics. As this works for use in terms of orientations.
In aeronautics the system NED is commmonly used to provide orientation with perpective of the vehicle. Where the X is pointing at the fore, the Y to starboard and finally the Z to the keel of the vehicle.
Now lets use this perspective to analyze the forces applied to the vehicle. Where the rotational velocity of the vehicle is measured by the total of three different forces.
The physical forces around the fore or X axis (lets call this force 'p'), the forces around Y  (lets call this force 'q') and lastly the forces around Z axis (lets call this force 'r').
Now the velocities around this same axis that we mentioned before will be: u, v and w (This variables are just being used as commonly used in control models).
With these now we can describe the variables that will be used for the euler angles. Where the orientation of a local system related to a global can be used, such as roll, pitch and roll.

\section{Mathematical Model}
The quadcopter is modeled as a rigid body with six degrees of freedom (6 DOF):
three translational movements along the x, y, and z axes, and three rotational movements around the x, y, and z axes (roll, pitch, and yaw).
However we say that the mathematical model of the quadcopter is not optimal.
The reason behind this is of the outputs of the system versus the input of the same system.
As an input, the system can only rely on four forces that will change the behaviour of the system, but with those four forces the system has six different results.
Even from the change of a single force.
The dynamics are described using Newton-Euler formalism.

\subsection{Coordinate Systems}
Two coordinate systems are used: the inertial frame (fixed) and the body frame (attached to the quadrotor). The transformation between these frames is described using rotation matrices.

\[
    R(\phi, \theta, \psi) = R_z(\psi)R_y(\theta)R_x(\phi)
\]

Where:
\[
    R_z(\psi) = \begin{bmatrix}
        \cos \psi & -\sin \psi & 0 \\
        \sin \psi & \cos \psi  & 0 \\
        0         & 0          & 1
    \end{bmatrix}, \quad
    R_y(\theta) = \begin{bmatrix}
        \cos \theta  & 0 & \sin \theta \\
        0            & 1 & 0           \\
        -\sin \theta & 0 & \cos \theta
    \end{bmatrix}, \quad
    R_x(\phi) = \begin{bmatrix}
        1 & 0         & 0          \\
        0 & \cos \phi & -\sin \phi \\
        0 & \sin \phi & \cos \phi
    \end{bmatrix}
\]

\section{Translational Dynamics}
The translational dynamics describe how the quadrotor moves along the x, y, and z axes. These dynamics depend on the thrust generated by the rotors and the orientation of the quadrotor.

\subsection{3 DOF Translational Dynamics}
In the 3 DOF model, the translational dynamics only consider movement in the \(x\) and \(y\) directions, assuming a constant altitude \(z\). The equations are:

\[
    \ddot{x} = \frac{F_t}{m} \sin(\theta)
\]
\[
    \ddot{y} = \frac{F_t}{m} \sin(\phi)
\]

Here, the pitch angle \(\theta\) influences the motion in the \(x\)-direction, and the roll angle \(\phi\) influences the motion in the \(y\)-direction.

\subsection{6 DOF Translational Dynamics}
In the 6 DOF model, the translational dynamics include movement in all three spatial dimensions (\(x\), \(y\), and \(z\)). The full equations consider the effects of pitch, roll, and yaw on the drone's movement:

\[
    \ddot{x} = \frac{F_t}{m} (\cos\phi \sin\theta \cos\psi + \sin\phi \sin\psi)
\]
\[
    \ddot{y} = \frac{F_t}{m} (\cos\phi \sin\theta \sin\psi - \sin\phi \cos\psi)
\]
\[
    \ddot{z} = \frac{F_t}{m} (\cos\phi \cos\theta) - g
\]

In the 6 DOF model, the pitch \(\theta\) and roll \(\phi\) angles still influence the motion in the \(x\) and \(y\) directions, but now these are coupled with the yaw angle \(\psi\), and the \(z\)-axis dynamics are explicitly considered.

\section{Rotational Dynamics}
The rotational dynamics describe how the quadrotor rotates around the x, y, and z axes. These dynamics depend on the torques generated by the rotors.

\subsection{3 DOF Rotational Dynamics}
The 3 DOF model simplifies the rotational dynamics by only considering rotation around the \(z\)-axis (yaw). The dynamics are given by:

\[
    \tau_z = I_z \ddot{\psi}
\]

Here, the yaw torque \(\tau_z\) is the only rotational dynamic considered, ignoring roll and pitch.

\subsection{6 DOF Rotational Dynamics}
The 6 DOF model incorporates full rotational dynamics, including roll (\(\phi\)), pitch (\(\theta\)), and yaw (\(\psi\)). The equations are:

\[
    \tau_x = I_x \dot{p} + (I_z - I_y) qr + I_{xz} (q^2 - r^2)
\]
\[
    \tau_y = I_y \dot{q} + (I_x - I_z) rp + I_{xz} (r^2 - p^2)
\]
\[
    \tau_z = I_z \dot{r} + (I_y - I_x) pq + I_{xz} (p^2 - q^2)
\]

These equations describe how torques around all three axes are influenced by the angular velocities and the moments of inertia.

\section{Linking 3 DOF and 6 DOF Dynamics}
The 3 DOF model is a simplified version of the 6 DOF model, where certain dynamics are ignored or assumed constant to reduce complexity.

\subsection{Translational Dynamics}
The 3 DOF translational dynamics are a subset of the 6 DOF translational dynamics, with:
\begin{itemize}
    \item The \(z\)-axis dynamics ignored (constant altitude assumption).
    \item The \(x\) and \(y\) dynamics simplified by assuming no coupling with the yaw angle \(\psi\).
\end{itemize}

\subsection{Rotational Dynamics}
The 3 DOF rotational dynamics are a special case of the 6 DOF rotational dynamics, where:
\begin{itemize}
    \item Only yaw (\(\psi\)) rotation is considered, while roll (\(\phi\)) and pitch (\(\theta\)) are assumed to be either zero or constant.
\end{itemize}

In summary, the 3 DOF model can be seen as a subset or simplification of the 6 DOF model, where certain dynamics are ignored or assumed constant to reduce complexity. When moving from 3 DOF to 6 DOF, you add more degrees of freedom, which results in more complex interactions between the forces and torques acting on the drone.

\chapter{State-Space Representation}

\section{3 DOF State-Space Representation}
To convert the equations of motion from the 3 DOF model into a state-space representation, we define the state vector \(\mathbf{x}\) and the input vector \(\mathbf{u}\). The state-space equations take the form:

\[
    \dot{\mathbf{x}} = A\mathbf{x} + B\mathbf{u}
\]
\[
    \mathbf{y} = C\mathbf{x} + D\mathbf{u}
\]

Where:
\[
    \mathbf{x} = \begin{bmatrix} x \\ y \\ \psi \\ \dot{x} \\ \dot{y} \\ \dot{\psi} \end{bmatrix}, \quad \mathbf{u} = \begin{bmatrix} u_x \\ u_y \\ u_\psi \end{bmatrix}
\]

The matrices \(A\), \(B\), \(C\), and \(D\) are derived based on the equations of motion.

\section{6 DOF State-Space Representation}
For the 6 DOF model, the state vector \(\mathbf{x}\) includes all six degrees of freedom (position and orientation, as well as their derivatives). The state-space equations are extended to include these dynamics:

\[
    \dot{\mathbf{x}} = A\mathbf{x} + B\mathbf{u}
\]
\[
    \mathbf{y} = C\mathbf{x} + D\mathbf{u}
\]

Where:
\[
    \mathbf{x} = \begin{bmatrix} x \\ y \\ z \\ \phi \\ \theta \\ \psi \\ \dot{x} \\ \dot{y} \\ \dot{z} \\ p \\ q \\ r \end{bmatrix}, \quad \mathbf{u} = \begin{bmatrix} u_x \\ u_y \\ u_z \\ u_\phi \\ u_\theta \\ u_\psi \end{bmatrix}
\]

Again, the matrices \(A\), \(B\), \(C\), and \(D\) are derived from the equations of motion, accounting for all six degrees of freedom.

\chapter{Conclusion}

\section{Summary}
This document has presented a comprehensive analysis of the quadrotor's dynamics in both 3DOF and 6 DOF models. The 3 DOF model provides a simplified view of the quadrotor's motion, focusing on horizontal movements and yaw rotation, while the 6 DOF model captures the full complexity of the quadrotor's dynamics, including vertical motion, roll, and pitch.

\section{Future Work}
Further work could include extending the analysis to consider aerodynamic effects, incorporating sensor models (such as IMUs) into the simulation, and validating the control strategies in real-world experiments. Additionally, exploring advanced control strategies such as adaptive or robust control could provide better performance in the presence of uncertainties or disturbances.

\end{document}
